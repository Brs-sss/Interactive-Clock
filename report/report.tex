\documentclass[UTF8]{ctexart}
\usepackage{amsmath}
\usepackage{multirow}
\usepackage{float}
\usepackage{graphicx}
\usepackage{geometry}
\usepackage{diagbox}
\pagestyle{plain}

\title{可交互式钟表页面说明文档}
\author{组长:白润声 2021011793 \\ 组员:王皓雯 2021011815、林敏芝 2021011791、郭嘉伟 2021011787}
\date{}

\CTEXsetup[name={第,部分}]{section}
\CTEXsetup[format={\zihao{-3}\raggedright\bfseries}]{section}
\geometry{a4paper,left=20mm,right=20mm}

\begin{document}
	\maketitle
	\section{实现思路} % 简要阐释功能点的实现思路
	% 以下是各功能点,大家可以简要解释自己负责部分的思路。文字写在每个\paragraph下的空行即可。
	\paragraph{逻辑架构}
	
	\paragraph{页面切换}
	
	\paragraph{钟表}
	
	\paragraph{表针拖动}
	
	\paragraph{计时器}
	
	\paragraph{秒表}
	
	\paragraph{闹钟}
	
	% -------------------- 分界线 -------------------- %
	
	\section{交互方式} % 简要说明各部分的使用方法
	\paragraph{整体页面}
	
	\paragraph{钟表}
	
	\paragraph{计时器}
	
	\paragraph{秒表}
	
	\paragraph{闹钟}
	

	
	\section{参考资料} % 如果有的话就写一下


\end{document}