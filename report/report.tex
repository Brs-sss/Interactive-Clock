\documentclass[UTF8]{ctexart}
\usepackage{amsmath}
\usepackage{multirow}
\usepackage{float}
\usepackage{graphicx}
\usepackage{geometry}
\usepackage{diagbox}
\usepackage{hyperref}

\pagestyle{plain}

\title{可交互式钟表页面说明文档}
\author{组长:白润声 2021011793 \\ 组员:王皓雯 2021011815、林敏芝 2021011791、郭嘉伟 2021011787}
\date{}

\CTEXsetup[name={第,部分}]{section}
\CTEXsetup[format={\zihao{-3}\raggedright\bfseries}]{section}
\geometry{a4paper,left=20mm,right=20mm}

\begin{document}
	\maketitle
	\section{实现结果}
	(可以截图放这,或者不要这部分也行)
	\section{实现思路} % 简要阐释功能点的实现思路
	% 以下是各功能点,大家可以简要解释自己负责部分的思路。文字写在每个\paragraph下的空行即可。
	\paragraph{逻辑架构}
	我们将时间离散化,每秒分为20tick,因此一个时刻对应(hour, minute, second, tick)。在此基础上我们实现了clock类,该类是所有表的逻辑架构,内部元素包含确定时刻的四个值,同时支持以下接口:1.获取时/分/秒针角度;2.获取时间字符串;3.获取与设置总tick数;3.设置时/分/秒/tick;4.tick自增与自减。每个不同的表都由一个clock类控制。
	\paragraph{页面切换}
	
	\paragraph{钟表}
	
	\paragraph{表针拖动}
	表针拖动功能的实现主要依靠表针对象对鼠标事件的监听。当鼠标拖动时,首先记录当前时间,之后算出鼠标移动的角度。根据不同的表针,算出不同的时间增量,再对钟表的时间进行更新。通过时间的更新,实现表针位置的自动更新。
	
	\paragraph{计时器}
	表盘绘制:利用svg技术,画出圆盘背景,指针点和标记点。表盘中动画由红、蓝两道条纹进行不断循环的animation实现。
	运行:基于clock类实现。利用setTimeInteval,每50ms控制clock自增,并获取秒针角度更新指针点、获取时间字符串以更新现实中。
	\paragraph{秒表}
	表盘绘制:利用svg技术,画出背景板,计时凹槽、环和点。
	运行:基于clock类实现。在时间设置区获取到时间后首先设置clock时间,记录总tick数。其后setTimeInteval,每50ms控制clock自减,通过剩余总tick计算剩余比例以设置计时环和点。页面下方的设置时间区和按钮区为两大小、位置相同的区域,交替设置visible、hidden以实现切换效果。
	\paragraph{闹钟}
	
	% -------------------- 分界线 -------------------- %
	
	\section{交互方式} % 简要说明各部分的使用方法
	\paragraph{整体页面}
	整体页面包括导航栏与主体页面。导航栏由4个图标与1个marker组成。4个图标对应的功能分别为钟表、计时器、秒表和闹钟。当鼠标悬浮或点击图标时,对应的图标会亮起,并且marker会平滑地移至图标后侧,同时呈现出不同色彩。点击不同图标,页面会平滑地进行切换。
	\paragraph{钟表}
	拖动钟表的表针即可实时改变钟表的时间。

	\paragraph{计时器}
	左下侧为开始/暂停按钮,右下侧为重置按钮。点击开始按钮,计时器开始运行,其后切换为暂停按钮,再次点击则计时器暂停。重置按钮仅在暂停时可用,点击后会将时间设置为00:00:00。
	\paragraph{秒表}
	在时、分、秒框中输入时间后,若输入合法则会开始计时,且输入框切换为控制按钮。控制按钮依次为开始/暂停按钮、重置按钮和取消按钮。其中重置按钮点击后会重新将时间恢复为最初设置时间而,取消按钮点击后则会清除当前计时并切换回输入框。
	\paragraph{闹钟}
	在时、分、秒框中输入时间并在闹钟名称部分输入闹钟名后,若输入合法且闹钟名称不为空则设置闹钟成功,会在输入区下方累计显示闹钟的名称和时间,并分别显示每个闹钟的删除按钮,单击删除按钮可以删除对应闹钟。当时间达到闹钟设置的时间时,就会弹出到了闹钟时间的提示,并播放闹铃。
	

	
	\section{参考资料} % 如果有的话就写一下
	JS + HTML + CSS 实现Todolist-CSDN博客: \url{https://blog.csdn.net/Wksycxy/article/details/127792977?ops_request_misc=&request_id=&biz_id=102&utm_term=todolist}

	
	【教程演示】2021最具创意的16个CSS特效:\url{https://www.bilibili.com/video/BV1nT4y1977i?p=2&vd_source=873c347266938fb6766020bedd9fb369}

\end{document}